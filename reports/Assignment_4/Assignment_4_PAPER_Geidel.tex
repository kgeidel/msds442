\documentclass[11pt,letterpaper]{article}

%\usepackage{times}
%\usepackage{epsfig}
\usepackage{graphicx}
%\usepackage{amsmath}
%\usepackage{amssymb}
\usepackage{hyperref}
\usepackage{tabularx}
\usepackage[left=1in, right=1in, top=1in, bottom=1in]{geometry}
\usepackage{titling}
\usepackage{setspace}
\usepackage{sectsty}
\usepackage{tabto}
\graphicspath{{./Assignment_04_NOTEBOOK_Geidel_files/}} % adds the assets directory to the path, throw your images there
\usepackage{fancyhdr}
\pagestyle{fancy}
\fancyhf{}
\fancyheadoffset{0cm}
\renewcommand{\headrulewidth}{0pt} 
\renewcommand{\footrulewidth}{0pt}
\fancyhead[R]{\thepage}
\fancypagestyle{plain}{%
  \fancyhf{}%
  \fancyhead[R]{\thepage}%
}

\usepackage{cite}
\usepackage[sectionbib]{natbib}
\renewcommand{\refname}{}

\begin{document}
\fontfamily{ptm}\selectfont
\sectionfont{\fontsize{12}{12}\fontfamily{ptm}\selectfont}
\doublespacing
%%%%%%%%%%%%%%%%%%%%%%%%%%%%%% TITLE %%%%%%%%%%%%%%%%%%%%%%%%%%%%%%%%%%%%%%
\setlength{\droptitle}{1in} 

\title{\large{ASSIGNMENT 4: \\ NORTHWESTERN MEMORIAL HEALTHCARE AGENT \\\vspace{1.2in}}}

\author{
Kevin Geidel \\
MSDS 442: AI Agent Design \& Development \\
Northwestern University \\
May 25, 2025 \\
}

\date{}
\maketitle
\thispagestyle{empty}	
\clearpage
\setcounter{page}{1}

%%%%%%%%%%%%%%%%%%%%%%%%%%%%%% PAGE 1 %%%%%%%%%%%%%%%%%%%%%%%%%%%%%%%%%%%%

% \section*{Requirement 1: Graph the agent with LangChain/LangGraph}
% \tab The construction of the agent and accompanying graph begins with the creation of the functions that serve as edges in our graph (see cell 3 in the appendix). 
% The actual assembly of the graph itself occurs in cell 9. However, some of the components, such as the conditional edge, \texttt{verify\_policy}, and \textbf{ClaimState} class,
% are built above. Following the logic in cell 9 we first instantiate an empty graph:

% \begin{verbatim}
% workflow = StateGraph(AgentState)
% \end{verbatim}

% The \texttt{workflow} object has \texttt{add\_node} and \texttt{add\_edge} methods that allow us to assemble the 
% components created in cells 3-8. The output is displayed graphically in cell 10 (reproduced in figure \ref{fig:graph} below.)

Link to the Grok session for knowledge base generation: \url{https://grok.com/share/bGVnYWN5_2b101ac8-5575-458e-a8be-7c50e3134e34}

\end{document}